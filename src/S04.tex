\documentclass[
    accentcolor=pink,
    boxarc,
    dark_mode,
    logofile=enmpty
]{rubos-tuda-template}

\usepackage{rubos-common}
\usepackage{wasysym}

\sheetnumber{4}
\semester{SoSe 2023}
\date{11. Mai 2023}
\termStyle{left-right-manual}
\termLeft{%
    printSemester,%
}
\termRight{%
    printDate,%
}
\graphicspath{{../pictures/}}

\title[trans rights <3]{Treffpunkt Mathematik II für Informatik \\ Sitzung \getSheetnumber{}}

\begin{document}
    \maketitle{}

    \begin{anmerkung}
        \huge{\textcolor{pink}{Keine Garantien über Richtigkeit oder Vollständigkeit. \\ Dies ist ein freiwilliger Mitschrieb.}}
    \end{anmerkung}

    Grenzwerte von Funktionen:
    \[\lim_{x \rightarrow x^*} f(x) = f^* \equiv \text{Für jede Folge $(x_n)_{n \in \mathbb{N}}$ mit $x_n \overset{n \rightarrow \infty}{\longrightarrow} x^*$ gilt} \lim_{n \rightarrow \infty} f(x_n) = f^*\]
    \[ x_n = x^* + \frac{1}{n} \ \ y_n = x^*+\frac{1}{n^2}\]
    Stetigkeit von Funktionen:
    \[f \text{ ist stetig im Punkt } x_0 \equiv \lim_{x \rightarrow x_0}f(x) = f(x_0)\]
    Häufungspunkte einer Menge D:
    \[x_0 \in D \text{ Häufungspunkt von D } \equiv \exists (x_n) \subset D \text{ mit } x_n \overset{n \rightarrow \infty}{\longrightarrow} x_0 \text{ und } x_n \ne x_0\]

    \subsection*{Aufgabe 4.1}
    \begin{enumerate}
        \item[i)]
            $\lim_{x \rightarrow 0} \sin(\frac{1}{x})$ Überprüfe, ob ein $\alpha \in \mathbb{R}$, sodass $\lim_{x \rightarrow 0} \sin(\frac{1}{x}) = \alpha$\\
            Müssen uns anschauen $\lim_{n \rightarrow \infty} \sin(\frac{1}{x_n})$ mit $x_n \overset{n \rightarrow \infty}{\longrightarrow} 0$\\
            Reziproke(??): $\overset{\sim}{x_n} := \frac{\pi}{2} +n2\pi \Rightarrow \sin(\overset{\sim}{x_n}) = 1\ \forall n \in \mathbb{N}$
            \[\leadsto x_n = (\overset{\sim}{x_n})^{-1} \Rightarrow \sin(\frac{1}{x_n}) = \sin(\frac{1}{(\overset{\sim}{x_n})^{-1}}) = \sin(\overset{\sim}{x_n}) = 1\]
            \[\leadsto \overset{\sim}{y_n} := \pi + n2\pi\ \ y_n := (\overset{\sim}{y_n})^{-1} \Rightarrow \sin\frac{1}{y_n} = \sin(\frac{1}{(\overset{\sim}{y_n})^{-1}}) = sin(\overset{\sim}{y_n}) = 0\ \text{\lightning}\]
        \item[ii)]
            \[\lim_{x \rightarrow 0} x\sin(\frac{1}{x}) \overset{z. z.}{=} 0\]
            $\leadsto$ Sei $(x_n)_n$ Nullfolge $|x_n\sin(x_n)| \le |x_n|$
    \end{enumerate}
    \subsection*{Aufgabe 4.2}
    \begin{anmerkung}
        Wurde nicht bearbeitet.
    \end{anmerkung}

    \subsection*{Aufgabe 4.3}
    \begin{anmerkung}
        Wurde nicht bearbeitet.
    \end{anmerkung}

    \subsection*{Aufgabe 4.4}
    \begin{enumerate}[label={\alph*)}]
        \item
            \[f: \mathbb{R}\backslash\{0\} \rightarrow \mathbb{R}, x \mapsto 2 - \frac{|x|}{x}\]
            Frage: Gibt es $f^*\mathbb{R}$, sodass $\overset{\sim}{f}(x) =\begin{cases}
                    f(x), \text{ für } x \ne 0 \\
                    f^*, \text{ sonst}
                \end{cases}$ stetig auf ganz $\mathbb{R}$ ist? \\
            Antwort: Angenommen $f$ stetig fortsetzbar in 0. $f^* \overset{!}{=} \lim_{x \rightarrow 0^-} \overset{\sim}{f}(x) = 3 \ne 1 = \lim_{x \rightarrow 0^+} \overset{!}{=} f^*\ \text{\lightning}$ \\
            Kommentar von mir: $f(x) = \begin{cases}
                    3, \text{ falls } x < 0 \\
                    1, \text{ sonst}
                \end{cases}$, das wurde angezeichnet
        \item
    \end{enumerate}

    \subsection*{Aufgabe 4.5}
    \[f(x) = \frac{1}{9}x^2 + \frac{1}{5}x + 1, x \in [1, 2]\]
    Zu zeigen: $f$ besitzt in $f(1,2)$ genau einen FP, i.e. $\exists!\ x^* \in (1,2):f(x^*) = x^*$\\
    Banachscher Fixpunktsatz $\leadsto$ z.z.:
    \begin{enumerate}
        \item[i)]
            $f$ ist Selbstabbildung, i.e. $f([1,2]) \subseteq [1,2]$

        \item[ii)] $\exists\ q < 1 : |f(x)-f(y)| \le q|x-y|\ \forall\ x,y \in [1,2] = I$
    \end{enumerate}
    Beweis:
    \begin{enumerate}
        \item[i)]
            \[f(x) = \frac{1}{9}x^2 + \frac{1}{5}x + 1 \le \frac{4}{9} + \frac{2}{5} + 1 = \frac{20+18}{45} + 1 \le 2\ \forall\ x \in I\]
            \[f(x) \ge \frac{1}{9} + \frac{1}{5} + 1 \ge 1\]

        \item[ii)]
            \[|f(x) - f(y)| = |\underbrace{\frac{1}{9}(x^2+y^2)}_{\dagger}+\frac{1}{5}(x-y)| \le (\frac{4}{9} + \frac{1}{5})|x-y|\ \forall\ x,y \in I\]
            \[\dagger: \frac{1}{9}|(x+y)(x+y)| \le \frac{4}{9} |x+y|\ \forall\ x,y \in I\]

    \end{enumerate}
\end{document}
