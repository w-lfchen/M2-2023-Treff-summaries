\documentclass[
    accentcolor=pink,
    boxarc,
    dark_mode,
    logofile=enmpty
]{rubos-tuda-template}

\usepackage{rubos-common}

\sheetnumber{9}
\semester{SoSe 2023}
\date{13. Juni 2023}
\termStyle{left-right-manual}
\termLeft{%
    printSemester,%
}
\termRight{%
    printDate,%
}
\graphicspath{{../pictures/}}

\title[trans rights <3]{Treffpunkt Mathematik II für Informatik \\ Sitzung \getSheetnumber{}}

\begin{document}
    \maketitle{}
    \begin{anmerkung}
        \huge{\textcolor{pink}{Keine Garantien über Richtigkeit oder Vollständigkeit. \\ Dies ist ein freiwilliger Mitschrieb.}}
    \end{anmerkung}

    \subsection*{9.1}
    \[a < b,\quad f:[a,b] \to \mathbb{R} \text{ stetig}\]
    \begin{anmerkungen}
        Aus der Stetigkeit von $f$ folgt, dass $f$ integrierbar ist.\\
        Aus der Stetigkeit von $f$ folgt, dass $|f|$ stetig ist.
    \end{anmerkungen}

    \subsubsection*{i)}
    $|f|$ ist integrierbar und es gilt
    $\displaystyle{|\int_{a}^{b}f(x)dx| \le \int_{a}^{b}|f(x)|dx}$
    \begin{proof}
        Schritt 1: $|f| = f^+ + f^-$ integrierbar, da $|f|$ stetig
        \[\leadsto -f, f \le |f|\]
        \[\Rightarrow \int_{a}^{b}f(x)dx \le \int_{a}^{b}|f(x)|dx \text{ und } \int_{a}^{b}f(x)dx \le \int_{a}^{b} |f(x)|dx \text{ wegen Monotonie (und Homogenität)}\]
    \end{proof}
    \subsubsection*{Monotonie des Integrals}
    Seien $f$, $g$ integrierbar mit $f \le g$ auf $[a,b]$.
    \[\text{Zu zeigen: } \int_{a}^{b}f(x)dx \le \int_{a}^{b}g(x)dx\]
    \begin{proof}
        \begin{align*}
            \leadsto & R(f, Z) = \sum_{i=1}^{|Z|}f(\eta_i)\Delta_i  \\
                     & R(g, Z) = \sum_{i=1}^{|Z|}g(\eta_ii)\Delta_i
        \end{align*}
    \end{proof}
    \begin{anmerkungen}
        $R(f, Z)$ ist Riemann-Integral mit Funktion $f$ und Partitionierung $Z$\\
        Von mir: hierbei die Rechtecke unter der Funktion im Hinterkopf behalten
    \end{anmerkungen}

    \subsubsection*{ii)}
    Sei $c \in (a,b)$. Dann ist $f$ auf $[a,c]$ und $[c,b]$ integrierbar und es gilt $\displaystyle{\int_{a}^{b}f(x)dx = \int_{a}^{c}f(x)dx + \int_{c}^{b}f(x)dx}$
    \begin{proof}
        Nehme Partitionierung $Z^j = (z_1^j, \dots, z_i^j, \dots ,z_n^j)$(Anmerkung: $z_1^j = a, z_n^j = b$) mit $z_i^j = c$. Dann gilt
        \[R(f, Z^j) = \sum_{k=1}^{n}f(z_k^j)\Delta_k^j = \sum_{k=1}^{i}f(z_k^j) \Delta_k^j + \sum_{k=i}^{n}f(z_n^j)\Delta_k^j\]
        \begin{anmerkungen}
            Das $j$ indiziert die Partitionierung, "bestimmt die Feinheit der Partitionierung"\\
            Beim Integralausdruck kann man über Riemann-Summen argumentieren, wie es hier getan wurde.
        \end{anmerkungen}
        $\leadsto$ nach Grenzübergang folgt wzzw (was zu zeigen war)
    \end{proof}

    \subsection*{9.2}
    Wir betrachten das Skalarprodukt auf $C([a,b])$ (Raum der stetigen Funktionen), i.e.
    \[\langle f,g \rangle := \int_{a}^{b}f(x)g(x)dx \quad \text{für } f,g \in C([a.b])\]
    und die dadurch induzierte Norm $\displaystyle{||f||_2 = \int_{a}^{b}f(x)^2dx}$.

    Zu zeigen: $||\cdot||_2$ ist nicht äquivalent zur Supremumsnorm $||f||_\infty = \underset{[a,b]}{\sup}|f|$
    \begin{proof}
        Äquivalenz der Normen $\sim c||f||_\infty\le ||f||_2\le C||f||_2$
        \[\leadsto f_n(x) = \begin{cases}
                \sqrt{n} & \text{für }x \in [0, \frac{1}{n}] \\
                0        & \text{sonst}
            \end{cases} \Rightarrow ||f_n||_2^2 = \int_{0}^{\frac{1}{n}}ndx=1 \text{ und } ||f_n||_\infty = \sqrt{n}\]
        Der Beweis ist als Hausübung zu Ende zu führen.
    \end{proof}
\end{document}
